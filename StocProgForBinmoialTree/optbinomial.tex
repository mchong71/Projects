\documentclass[12pt]{article}
\usepackage{sydewkrpt}
\usepackage{longtable}
\usepackage{array}
\usepackage{ragged2e}
\newcolumntype{P}[1]{>{\RaggedRight\hspace{0pt}}p{#1}}

%%%%%%%%%%%%%%%%%%%%%%%%%%%%
%%%    Begin Document    %%%
%%%%%%%%%%%%%%%%%%%%%%%%%%%%
\begin{document}
\pagenumbering{roman}

\waterlootitle{Two-Stage Stochastic Programming Option Portfolio Using Binomial Tree Pricing\\}{
  SYDE 531: Final Project
}{
  D. Scott Neil -- 20349210\\
  Matthew Chong -- 20341648\\
}

%\dotableofcontents

\newpage 
\doublespacing
\pagenumbering{arabic}
\setlength{\parindent}{1cm}

\section{Introduction}

\section{Objective}
To create an optimal portfolio of two independant options, priced using the Jarrow Rudd \ref{jarrow_rudd} binomial model, with the two-stage stochastic programming method.

\section{Assumptions}
The assumptions made are: the options mature in 4 months, the interest thing is 0.03, periods are measured as a month, other stuff

\section{Objectives and Requirements}

\section{Testing Methodology}

\section{Two Option Portfolio}

\section{n-option portfolio}
this can easily be extended to encompass a larger portfolio, but just requires more data

\section{Future Work}
% include stock and other securities in model to make a well balanced portfolio
% consider options over longer periods - using the real option data
% include more options in model - this doesn't change calculations, just increases computation

\section{Conclusion}


\newpage
\addcontentsline{toc}{section}{References}

\bibliographystyle{IEEEtran}

\bibliography{bib}

\end{document}
